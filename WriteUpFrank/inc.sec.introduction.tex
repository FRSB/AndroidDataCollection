\section{Einleitung}

\begin{description}
\item[Problemstellung] Heranf�hren an das Thema. Warum soll Arbeit geschrieben werden. Offene Frage mit Problembeschreibung.
\item[Zielstellung] Ein oder zwei S�tze zum Ziel.
\item[Methodik und Vorgehensweise] Wie werden Ergebnisse ermittelt? Gliederung der Arbeit in Worten.
\end{description}

Computer sind heutzutage allgegenw�rtig. Viele Menschen besitzen mobile Ger�te, die in Sachen Rechenleistung station�ren Computern nur noch in wenig nachstehen.

Der wirtschaftliche Nutzen des sogenannten �ubiquitous computing� ist vielseitig. So k�nnten beispielsweise Restaurantbesitzer besondere Mittagsangebote auf das Smartphone der potenziellen Kunden senden, die noch unentschlossen vor ihrem Lokal stehen. K�nnte man vorhersehen, wo sich die Anwender zuk�nftig hinbewegen, so k�nnten Angebote schon im Vorfeld unterbreitet oder Informationen verschickt werden.

Die Idee des Projekts ist es, eine Handyanwendung zu programmieren, die die n�tigen Daten f�r eine verl�ssliche Positionsvorhersage sammelt, und diese auszuwerten. Somit soll untersucht werden, ob es prinzipiell m�glich w�re (und mit welcher Genauigkeit), Bewegungen von Menschen vorherzusagen.
