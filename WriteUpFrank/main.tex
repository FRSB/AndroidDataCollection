\documentclass[a4paper, 12pt, titlepage, bibtotoc]{scrartcl} %scrartcl f�r kurze Artikel

\renewcommand*\sectfont{\normalcolor\rmfamily\bfseries}
\renewcommand*\descfont{\rmfamily\bfseries}
\setkomafont{dictum}{\normalfont\normalcolor\rmfamily\small}

\makeatletter% siehe De-TeX-FAQ
\renewcommand*{\toc@heading}{%
  \addsec{\contentsname}% bei scrartcl \addsec statt \addchap
  \@mkboth{\contentsname}{\contentsname}%
}
\makeatother% siehe \makeatletter

\usepackage[ngerman]{babel}
\usepackage[ansinew]{inputenc}
\usepackage[T1]{fontenc}
\usepackage{color}
% \usepackage{courier} % Monospace/Truetype umstellen
\usepackage{enumitem} % \begin{enumerate}[label={\alph*)}] \item \end{enumerate} f�r a) b) c)...
\usepackage{amsmath}
\usepackage{amssymb}
\usepackage{setspace}
\usepackage{ulem}
	\normalem %stellt emph{} wieder auf kursiv, sonst unterstrichen
\usepackage{graphicx}
\graphicspath{{img/}}
\usepackage[hmargin={3cm,3cm}, vmargin={2.5cm, 3.5cm}]{geometry}
% \usepackage{times}
\usepackage{mathptmx}
\usepackage{nicefrac} 	
\usepackage{titlesec}
\titlespacing{\section}{0pt}{*2.5}{*1.5}
\titlespacing{\subsection}{0pt}{*2}{*1}
\titlespacing{\subsubsection}{0pt}{*1}{*0.5}

\tolerance=500

\usepackage{url}
\urlstyle{leo}

% Punkt + Komma Abst�nde bei Tausendern/Dezimalzahlen ans dt. anpassen
% \mathcode`,="013B
% \mathcode`.="613A

%\usepackage{hyperref}
\usepackage{csquotes}
\usepackage{parcolumns}
% \usepackage{parskip}

\bibliography{recommendation-in-social-networks_literature}
\usepackage[style=authoryear]{biblatex} 
\addbibresource{recommendation-in-social-networks_literature.bib}

\DefineBibliographyStrings{ngerman}{
    references = {Literaturverzeichnis}
}

\defbibenvironment{bibliography}
{\list{}
{\setlength{\leftmargin}{\bibhang}%
\setlength{\itemindent}{-\leftmargin}%
\setlength{\itemsep}{12px}%
\setlength{\parsep}{\bibparsep}}}
{\endlist}
{\item}

% \renewbibmacro*{cite:year+labelyear}{%
% \iffieldundef{year}
% {}
% {\printtext[bibhyperref]{%
% \mkbibparens{%
% \printfield{year}%
% \printfield{labelyear}}}}}


%%%% Klammern um die Jahreszahl 
% \renewbibmacro*{cite:labelyear+extrayear}{%
% \iffieldundef{labelyear}
% {}
% {\printtext[bibhyperref]{%
% \mkbibparens{\iffieldundef{origyear}{}{\printfield{origyear}\addslash}% <--- added
% \printfield{labelyear}%
% \printfield{extrayear}}}}}
% 
% \renewbibmacro*{date+extrayear}{%
% \iffieldundef{year}
% {}
% {\printtext[parens]{%
% \iffieldundef{origyear}{}{\printfield{origyear}\addslash}%<--- added
% \printdateextra}}}

%Schriftart http://tu-dresden.de/Members/jan.rudl/latex_win/fonts.pdf
%\usepackage[T1]{fontenc}
%\newcommand{\changefont}[3]{
%\fontfamily{#1} \fontseries{#2} \fontshape{#3} \selectfont}

% \setlength{\tabcolsep}{5pt} %colspan
% \renewcommand{\arraystretch}{1.3} %rowspan

\usepackage{parskip}

\begin{document}

\begin{titlepage}
\begin{center}

\ 

\vspace{\baselineskip}

Martin-Luther-Universit�t\\
Halle-Wittenberg

\vspace{\baselineskip}

-- Juristische und Wirtschaftswissenschaftliche Fakult�t --

\vspace{\baselineskip}

\includegraphics[width=5cm]{mlu}

\vspace{\baselineskip}

\textbf{Seminararbeit}

\textbf{\large{Auswerten von Bewegungsmustern mittels Mobiltelefonen}}

\vspace{\baselineskip}

\parbox{0cm}{
\begin{tabbing}
XXXXXXX\= \kill \\ 
Seminar:\> "`Seminar Information Systems und E-Business"'\\
\\
Dozenten:\> Prof. Dr. Ralf Peters\\
\> M.Sc. WI Uwe Bretschneider\\
\> Dipl. WI Sebastian K�hler\\
\> Dipl. WI Thomas W�hner
\end{tabbing}}

\vspace{3\baselineskip}

\parbox{0cm}{
\begin{tabbing}
XXXXXXXXXXXXXXXXX\= \kill \\
Frank Rosner\>2. Semester\\
Goethestra�e 3\> Matr.-Nr. 209211038\\
06114 Halle\>Wirtschaftsinformatik\\
\\
Abgabetag:\>\today
\end{tabbing}
}

\end{center}
\end{titlepage}

% \setcounter{page}{0}\clearpage\newpage
\renewcommand{\baselinestretch}{1.25}\normalsize


\pagenumbering{Roman} \setcounter{page}{2}
\addcontentsline{toc}{section}{Inhaltsverzeichnis}
\tableofcontents

\renewcommand{\baselinestretch}{1.50}\normalsize

\newpage

\addcontentsline{toc}{section}{Abbildungsverzeichnis}
\listoffigures

\newpage

\section*{Abk�rzungsverzeichnis}
\addcontentsline{toc}{section}{Abk�rzungsverzeichnis}

\begin{tabular}{ll}
% CF & Collaborative Filtering\\
AA & Abk�rzungsaffe\\
AA & Abk�rzungsaffe
\end{tabular}

\newpage

\listoftables
\addcontentsline{toc}{section}{Tabellenverzeichnis}

\newpage

\pagenumbering{arabic} \setcounter{page}{1}

\section{Einleitung}

\begin{description}
\item[Problemstellung] Heranf�hren an das Thema. Warum soll Arbeit geschrieben werden. Offene Frage mit Problembeschreibung.
\item[Zielstellung] Ein oder zwei S�tze zum Ziel.
\item[Methodik und Vorgehensweise] Wie werden Ergebnisse ermittelt? Gliederung der Arbeit in Worten.
\end{description}

Computer sind heutzutage allgegenw�rtig. Viele Menschen besitzen mobile Ger�te, die in Sachen Rechenleistung station�ren Computern nur noch in wenig nachstehen. Der wirtschaftliche Nutzen des sogenannten "`ubiquitous computing"' ist vielseitig. So k�nnten beispielsweise Restaurantbesitzer besondere Mittagsangebote auf das Smartphone der potenziellen Kunden senden, die noch unentschlossen vor ihrem Lokal stehen. K�nnte man vorhersehen, wo sich die Anwender zuk�nftig hinbewegen, so k�nnten Angebote schon im Vorfeld unterbreitet oder Informationen verschickt werden.

Ziel des Projekts ist es, eine Handyanwendung zu programmieren, welche die n�tigen Daten f�r eine verl�ssliche Positionsvorhersage sammelt, und diese auszuwerten. Somit soll untersucht werden, mit welcher Genauigkeit Bewegungen von Menschen vorhergesagt werden k�nnen. Im Rahmen dieser Arbeit liegt der Fokus auf der Positionsvorhersage.

Methodik: Java / AndroidApp, R

Related Work...

Im ersten Abschnitt ...

\section{Methoden}
\subsection{Datenerhebung}

\begin{itemize}
\item Android-App zeichnet Daten live auf (Prozess im Hintergrund)
\item Mobilfunkzelle (CID, LAC) und Positionskoordinaten alle $x$ Sekunden bestimmen
\item CSV speichern -> runterspielen -> importieren
\end{itemize}

Visualisierung

\begin{figure}[p]
\begin{center}
\includegraphics*[width=1\linewidth]{cell_locations}
\caption{Locations}
\label{img:locations}
\end{center}
\end{figure}

\begin{figure}[p]
\begin{center}
\includegraphics*[width=.9\linewidth]{map.png}
\caption{Karte}
\label{img:map}
\end{center}
\end{figure}

\subsection{Datenanalyse}

\paragraph{Modellbildung}

\begin{itemize}
\item Nutzer bewegt sich durch die Stadt und �ndert seine Position
\item Modellierung einer Bewegung als stochastischer Prozess
	\begin{itemize}
	\item Nutzer $u$ hat zum Zeitpunkt $t$ eine Position $x_t \in S$
	\item Beobachtungszeitraum $T = t_0, \ldots, t_n$
	\item Zustandsraum $S$ $\mathrel{\widehat{=}}$ Menge der Mobilfunkzellen
	\item Bewegung ist diskrete Zustands�nderung
	\end{itemize}
\end{itemize}

\paragraph{Markov-Ketten 1. Ordnung}

\begin{itemize}
\item Zustandsdiagramm
\item Graphisches Modell
\item ML-Sch�tzer
\end{itemize}

\paragraph{Markov-Ketten h�herer Ordnung}

\begin{itemize}
\item Unterschied zu 1. Ordnung
\item Sch�tzer (Herleitung im Anhang?)
\end{itemize}

\section{Auswertung}

\subsection{Vorhersagegenauigkeit}

\begin{itemize}
\item 
\end{itemize}

\section{Schlussbetrachtung}

\begin{description}
\item[Zusammenfassung] die wichtigsten Ergebnisse der Arbeit werden zusammengefasst
\item[Kritische W�rdigung und Ausblick] Erl�uterung der Schw�chen und Grenzen der (Methodik der) Arbeit. Durch welche Verbesserungen und weiteren Schritte kann diesen Schw�chen entgegengewirkt werden? Ggf. Ausblick auf weitere offene Forschungsfragen bez�glich der Arbeit.
\end{description} 



\newpage

\parskip 12pt %Absatzabstand
\renewcommand{\baselinestretch}{1.00}\normalsize

\printbibliography

\section*{Anhang}

\begin{itemize}
\item im folgenden am Beispiel Markov-Ketten erster Ordnung
\item Zeitdiskrete Markov-Kette
\item Zustandsraum $S \in \{1, 2, 3, \ldots\}$
\item homogene Markov-Kette: $p_{ij}(t)=p_{ij}$ (station�re �bergangswahrscheinlichkeiten)
\item $P = p_{ij}$ ist �bergangsmatrix ($p_{ij} \geq 0$, $\sum_{j_\in S} p_{ij} = 1$)
\item Stochastischer Prozess $X=\{X_t, t\in N_0\}$, $S$ diskret hei�t Markov-Kette $p$-ter Ordnung, gdw. $\forall i_0, i_1, \ldots, i_{t+1} \in S, t \geq p+1$ gilt\\
$P(X_{t+1} = i_{t+1} | X_t=i_t, \ldots, X_{t-p+1} = i_{t-p+1}, \ldots, X_0 = i_0)$\\
$= P(X_{t+1} = i_{t+1} | X_t=i_t, \ldots, X_{t-p+1} = i_{t-p+1})$ (Markov-Eigenschaft)
\item Inferenz bei homogenen Markov-Ketten erster Ordnung $\hat p_{ij} = \frac{n_{ij}}{n_{i}}$, mit $n_{ij}$ als Anzahl beobachteter �berg�nge von $i$ nach $j$.
\item \textbf{Maximum Likelihood erster Ordnung}
\begin{enumerate}
%\item $ \displaystyle L(p_0,P|X) = \prod_{t=1}^T L(x_t|x_{t-1}, P) \cdot L(x_0|p_0)$
%\item $ \displaystyle L(p_0,P) = P(X_0 = i_0, \ldots, X_T = i_T)$
%\item $ \displaystyle L(P) =p_{i_0}\cdot p_{i_0i_1}\cdot \ldots \cdot p_{i_{T-1}i_T}$
\item Likelihood:\\
$ \displaystyle L(P) = \prod_{i=1}^m\prod_{j=1}^m p_{ij}^{n_{ij}},\quad i,j \in \text{Zustandsraum}$
% Problem ist noch erster und letzter Zustand!
\item Log-Likelihood:\\
$ \displaystyle l(P) = \sum_{i=1}^m\sum_{j=1}^m n_{ij} \cdot \ln p_{ij},\quad s.t.\ \forall i: \sum_{j=1}^m p_{ij}=1$
\item $ \displaystyle \mathcal L(P, \lambda_1, \ldots, \lambda_m) = \sum_{i=1}^m \sum_{j=1}^m n_{ij} \ln(p_{ij}) - \sum_{i=1}^m \lambda_i \left(\left(\sum_{j=1}^m p_{ij}\right) - 1 \right)$
\item $ \displaystyle \frac{\partial \mathcal L}{\partial p_{ij}} = \frac{n_{ij}}{p_{ij}} - \lambda_i$
\item FOC: $ \displaystyle \frac{n_{ij}}{p_{ij}} - \lambda_i = 0\quad \Leftrightarrow \quad p_{ij} = \frac{n_{ij}}{\lambda_i}$
\item $ \displaystyle \frac{\partial \mathcal L}{\partial \lambda_{i}} = 1 - \sum_{j=1}^m p_{ij}$
\item FOC: $ \displaystyle 1 - \sum_{j=1}^m p_{ij} = 0$
\item $ \displaystyle 1 - \sum_{j=1}^m \frac{n_{ij}}{\lambda_i} = 0$
\item $ \displaystyle \sum_{j=1}^m n_{ij} = \lambda_i$
\item $ \displaystyle \hat p_{ij} = \frac{n_{ij}}{\sum_{j=1}^m n_{ij}}$
\end{enumerate}
\end{itemize}


\end{document}

