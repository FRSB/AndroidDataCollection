\documentclass[a4paper, 11pt, titlepage]{scrartcl} %scrartcl f�r kurze Artikel

% \renewcommand*\sectfont{\normalcolor\rmfamily\bfseries}
% \renewcommand*\descfont{\rmfamily\bfseries}
% \setkomafont{dictum}{\normalfont\normalcolor\rmfamily\small}

\makeatletter% siehe De-TeX-FAQ
\renewcommand*{\toc@heading}{%
  \addsec{\contentsname}% bei scrartcl \addsec statt \addchap
  \@mkboth{\contentsname}{\contentsname}%
}
\makeatother% siehe \makeatletter

\usepackage[ngerman]{babel}
\usepackage[ansinew]{inputenc}
\usepackage[T1]{fontenc}
\usepackage{lmodern}
% \usepackage{times}
% \usepackage{mathptmx}
\usepackage{color}
% \usepackage{courier} % Monospace/Truetype umstellen
\usepackage{enumitem} % \begin{enumerate}[label={\alph*)}] \item \end{enumerate} f�r a) b) c)...
\usepackage{rotating}
\usepackage{grffile}
\usepackage{amsmath}
\usepackage{amssymb}
\usepackage{subfig}
\usepackage{setspace}
\setcapindent{0em} 
\interfootnotelinepenalty=10000
\raggedbottom
\usepackage{algorithm}
\usepackage{algorithmic}
	\renewcommand{\algorithmiccomment}[1]{// #1}
\usepackage{ulem}
	\normalem %stellt emph{} wieder auf kursiv, sonst unterstrichen
\usepackage{graphicx}
\graphicspath{{img/}}
\usepackage[hmargin={3cm,2cm}, vmargin={2cm, 3cm}]{geometry}
\usepackage{nicefrac} 	
\usepackage{titlesec}
\titlespacing{\section}{0pt}{*2.5}{*1.5}
\titlespacing{\subsection}{0pt}{*2}{*1}
\titlespacing{\subsubsection}{0pt}{*1}{*0.5}
\bibliography{literature}
\usepackage[style=authoryear]{biblatex} 
\addbibresource{literature.bib}

\DefineBibliographyStrings{ngerman}{
    references = {Literaturverzeichnis}
}

\defbibenvironment{bibliography}
{\list{}
{\setlength{\leftmargin}{\bibhang}%
\setlength{\itemindent}{-\leftmargin}%
\setlength{\itemsep}{12px}%
\setlength{\parsep}{\bibparsep}}}
{\endlist}
{\item}
\ExecuteBibliographyOptions{dashed=false}

\tolerance=500

\usepackage[hyperref]{ntheorem}
\theoremstyle{plain}
\theorembodyfont{\upshape}
\theoremsymbol{\ensuremath{\ast}}
\theoremseparator{}
\newtheorem{Example}{Beispiel}
\newtheorem{Definition}{Definition}

\renewcommand{\topfraction}{0.7}	% max fraction of floats at top

\usepackage{url}
\urlstyle{leo}

%\usepackage{hyperref}
\usepackage{csquotes}
\usepackage{parcolumns}

\usepackage{parskip}

% \parindent 0pt %Absatzeinr�ckung
% \parskip 12pt %Absatzabstand

\begin{document}

\tableofcontents

\newpage

\section{Introduction}

\section{Related Work}

\section{Methods}

\subsection{Data Collection}

\subsection{Data Analysis}

\begin{enumerate}
%\item $ \displaystyle L(p_0,P|X) = \prod_{t=1}^T L(x_t|x_{t-1}, P) \cdot L(x_0|p_0)$
%\item $ \displaystyle L(p_0,P) = P(X_0 = i_0, \ldots, X_T = i_T)$
%\item $ \displaystyle L(P) =p_{i_0}\cdot p_{i_0i_1}\cdot \ldots \cdot p_{i_{T-1}i_T}$
\item Likelihood:\\
$ \displaystyle L(P) = \prod_{i=1}^m\prod_{j=1}^m p_{ij}^{n_{ij}},\quad i,j \in \text{Zustandsraum}$
% Problem ist noch erster und letzter Zustand!
\item Log-Likelihood:\\
$ \displaystyle l(P) = \sum_{i=1}^m\sum_{j=1}^m n_{ij} \cdot \ln p_{ij},\quad s.t.\ \forall i: \sum_{j=1}^m p_{ij}=1$
\item $ \displaystyle \mathcal L(P, \lambda_1, \ldots, \lambda_m) = \sum_{i=1}^m \sum_{j=1}^m n_{ij} \ln(p_{ij}) - \sum_{i=1}^m \lambda_i \left(\left(\sum_{j=1}^m p_{ij}\right) - 1 \right)$
\item $ \displaystyle \frac{\partial \mathcal L}{\partial p_{ij}} = \frac{n_{ij}}{p_{ij}} - \lambda_i$
\item FOC: $ \displaystyle \frac{n_{ij}}{p_{ij}} - \lambda_i = 0\quad \Leftrightarrow \quad p_{ij} = \frac{n_{ij}}{\lambda_i}$
\item $ \displaystyle \frac{\partial \mathcal L}{\partial \lambda_{i}} = 1 - \sum_{j=1}^m p_{ij}$
\item FOC: $ \displaystyle 1 - \sum_{j=1}^m p_{ij} = 0$
\item $ \displaystyle 1 - \sum_{j=1}^m \frac{n_{ij}}{\lambda_i} = 0$
\item $ \displaystyle \sum_{j=1}^m n_{ij} = \lambda_i$
\item $ \displaystyle \hat p_{ij} = \frac{n_{ij}}{\sum_{j=1}^m n_{ij}}$

\end{enumerate}

\section{Results}

\section{Discussion}



\printbibliography

% \bibliographystyle{babplain}
% \bibliography{literature}

\end{document}

